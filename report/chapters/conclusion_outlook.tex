% !TEX root = ../report.tex
\chapter{Conclusion and Outlook}

\section{Conclusion}
In this semester project, a novel map fusion approach was developed which uses multiple \acp{KFM} and skips \acp{KF} after each detected \ac{KFM}. With the skipping of \acp{KF}, the \acp{KFM} are spread over a bigger area and together with the usage of multiple \acp{KFM}, more information can be gathered and provided to the optimization procedures which results in a reduction of drift and an improved accuracy. The novel approach therefore outperforms the previous approach in terms of accuracy.\\

The second part of this semester project, the introduced culling approach removes redundant \acp{KF} and therefore the optimization procedures (\ac{PGO} and \ac{BA}) have to process less edges and vertices. This culling approach reduced the runtime significantly while maintaining the accuracy.\\

The evaluation of the \ac{LM} and the \ac{DL} algorithm for the optimization procedures has shown, that the best timing can be achieved if the \ac{PGO} uses the \ac{LM} algorithm and the \ac{BA} uses the \ac{DL} algorithm With this setting the timing of the system was further improved.\\

With the new map fusion approach, the culling of \acp{KF} and the usage of optimal optimization algorithms with respect to the runtime, an existing multi agent \ac{SLAM} system was improved which was the goal of this semester project.

\section{Limitations}
In the proposed map fusion approach, the number of \acp{KFM} and the number of \acp{KF} which are skipped after a \ac{KFM} depends on the data set. If a data set contains a big overlapping area between the map of the clients, the number of \acp{KFM} and the number of \ac{KF} skips can be chosen high and the map fusion approach will be able to fusee the maps with high accuracy.\\

If the same settings are now used with a data set, that contains only a small overlapping area, the map fusion approach will probably fail as it won't be able to find the required number of \acp{KFM}.\\

Contrary, if a setting with a small number of \acp{KFM} and \ac{KF} skips is used, the map fusion approach will be able to fuse the maps with both data sets but it might achieves a lower accuracy on the data set with a large overlap compared to a setting with a higher number of \acp{KFM} and \ac{KF} skips.

\section{Outlook}

The proposed map fusion approach takes the transformation of the first detected \ac{KFM} to align the two maps. A more sophisticated heuristic approach could be used to determine the transformation of the \acp{KFM} which will result in the best alignment.\\

In the implementation of the \ac{KF} culling only the first order neighbors (in the co-visibility graph) of the \acp{KF} of the \acp{KFM} are considered for the redundancy check. This could be adapted that higher order neighbors in the co-visibility graph are also taken into consideration. In this case, one has to be careful to set the redundancy threshold to a value that the optimization procedures can still be provided with enough information not to reduce the accuracy.

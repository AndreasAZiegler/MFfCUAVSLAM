\chapter*{Abstract}
\addcontentsline{toc}{chapter}{Abstract}
%\chapter*{Zusammenfassung}
%\addcontentsline{toc}{chapter}{Zusammenfassung}

\paragraph*{Motivation}
A correct and accurate common map is crucial for multiple robots to collaboratively performing tasks. 

\paragraph*{Problem statement}
In this semester project an existing multi agent \ac{SLAM} system should be extended to merge maps of single robots in a way that no false map alignment is guaranteed and that an optimal alignment is achieved by using multiple place matches. Also redundant information, a consequence of the merge of two maps, should be removed in order to get a good performance of the optimization routines e.g. \ac{BA}.

\paragraph*{Approach}
To achieve the first goal, a new approach is proposed which uses multiple \acp{KFM} to merge two maps. This proposed approach also use a novel idea to spread the \acp{KFM} over a bigger area by skipping \acp{KF} while detecting \acp{KFM}.\\
To remove redundant information \ac{KF} culling is performed after all \acp{KFM} were detected and before the main map merging. This way information which were present in both maps appear only once in the merged map and the \ac{PGO} and the \ac{BA} does not have unnecessary data to process.\\
To reduce the runtime of the optimization part (\ac{PGO} and \ac{BA}), the usage of Powell's dog leg non-linear least squares technique instead of the Levenberg-Marquardt optimization was evaluated and the system was adapted that the performance is increased.


\paragraph*{Result}
The proposed map merging approach reduces drift and achieves a better accuracy compared to the old approach. With the implemented \ac{KF} culling the number of \acp{KF} which the \ac{PGO} and the \ac{BA} have to process is decreased significantly and therefore a better timing is achieved. The usage of Powell's dog leg non-linear least squares technique reduced the runtime of the optimization furthermore.
